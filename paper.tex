\documentclass[10pt]{article}
\usepackage{fullpage}
\usepackage[pdftex]{graphicx}
\usepackage{rotating}
\usepackage{multirow}
\usepackage{subfigure}
\usepackage{url}
\usepackage{amsmath}
\usepackage{amssymb}
\usepackage{wasysym}
\usepackage{algorithm}
\usepackage{algpseudocode}

\newcommand{\etal}{\emph{et al.}}

\begin{document}
\title{K Nearest Neighbour Search Using Space-filling Curves}
\author{Camara Lerner \hspace{2cm} Christopher Rabl \\
  Department of Mathematics and Computer Science\\
  University of Lethbridge, Canada}

\maketitle

\begin{abstract}

\end{abstract}

\section{Introduction}


\section{Hilbert Curve}

The Hilbert Curve is a space filling curve that makes the next point in the curve only one unit distance from the previous point. The curve is defined based onthe amount of space the curve fills. The space that the curve fills can be in any dimensionally space $(n)$ with the length of each dimension filled equal to $2^m$. This means that for each point in the curve the maximum bit size for each dimension is $(m)$~\cite{Hamilton:2006}. 

The curve labels the points from $0$ to $2^{m*n} - 1$ in the order that the curve visits each point. The Hilbert Curve considered here will start in the bottom left hand corner and finishes in the upper left corner. 
 

\flushleft
Formula for Gray Code is in \ref{gray-code}.
\begin{equation}
  \label{gray-code}
  gc(i) = i \oplus \left( i \gg 1 \right) 
\end{equation}

Formula for $g$ is displayed in \ref{g}.
\begin{equation}
  \label{g}
  g(i) = \lfloor \log _{2} \left( gc(i) \oplus gc(i+1) \right) \rfloor + 1
\end{equation}


Formula for $d(i)$ is displayed in \ref{d}.
\begin{equation}
  \label{d}
  d(i) = 
    \begin{cases}
      0 & \quad \text{if $i = 0$}\\
      g(i - 1) \text{ mod } n & \quad \text{if $i$ is even} \\
      g(i) \text{ mod }  n & \quad \text{if $i$ is odd}
    \end{cases}
\end{equation}


Formula for the bitwise right rotation is displayed in \ref{right-rotate}. The bitwise left rotation is similar.
\begin{equation}
  \label{right-rotate}
  a \rightturn i = \left[ a_{n - 1 + i\text{ mod }n} \ldots a_{i\text{ mod }n} \right]_{\left[ 2 \right]}, \text{ where } a = \left[ a_{n-1} \ldots a_0\right]_{\left[ 2 \right]} 
\end{equation}


Formula for $e$ is displayed in \ref{e}.
\begin{equation}
  \label{e}
  e(i) =  
    \begin{cases}
      0 & \quad \text{if $i = 0$}\\
      e(i - 1) \oplus 2^{d(i-1)} \oplus 2^{g(i-1)}& \quad \text{otherwise} \\
    \end{cases}
\end{equation}



\begin{algorithm}
  \caption{The algorithm for calculating the Gray Code Inverse.     Given $g \in \mathbb{N}$, calculates the $i \in \mathbb{N}$ such that $gc\left( i \right) = g $ }
  \label{gray-code-inverse}
  \begin{algorithmic}[1]
    \Require $g \in \mathbb{N}$
    \Return $i \in \mathbb{N} \text{ such that } gc \left( i \right) = g $
    \State $ m \leftarrow \text{ number of bits to represent } g$ 
    \State $ \left( i, j \right) \leftarrow \left( g, 1 \right) $ 
    \While{$ j < m $ }
    \State $ i \leftarrow i \oplus \left( g \gg j \right)$ 
    \State $ j \leftarrow j + 1$ 
    \EndWhile
    % \oplus is exclusive or
    % \ll left shift
    % \gg right shift
    % \mid bitwise or
  \end{algorithmic}
\end{algorithm}

\begin{algorithm}
  \caption{Calculating the Hilbert index given any dimensional point ${\bf p}$ 
    of size $n$ as long as the bits used within an index of the hilbert curve 
    space is specified in ($m$).}
  \label{hilbert-point-to-index}
  \begin{algorithmic}[1]
    \Require $n, m \in \mathbb{N} - \{0\} \text{ and a point } {\bf p} \in \mathbb{N}^n$
    \Return $h \in \mathbb{N} \text{, the Hilbert index of the point }{\bf p}$
    \State $ \left( h, e, d \right) \leftarrow \left( 0, 0, 0 \right) $
    \For{$j\gets m - 1, 0$} 
    \State $ l \leftarrow \left[ \left( p_{n-1} , i \right) \ldots \left( p_0 , i \right) \right]_{\left[ 2 \right]} $ 
    \State $ l \leftarrow \left( b \oplus e \right) \rightturn \left( d+1 \right)$ 
    \State $ w \leftarrow gc^{-1} \left( l \right)$ 
    \State $ e \leftarrow e \oplus \left( e \left( w \right) \leftturn \left( d+ 1 \right) \right) $ 
    \State $ d \leftarrow d + d \left( w \right) + 1 \text{ mod }n$ 
    \State $ h \leftarrow \left( h \ll n \right) | w $
    \EndFor
  \end{algorithmic}
\end{algorithm}

\bibliographystyle{amsplain}
\bibliography{mybiblio}

\end{document}

